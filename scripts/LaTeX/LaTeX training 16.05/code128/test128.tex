% test128.tex -- A file for testing code128 macro
%%%%%%%%%%%%%%%%%%%%%%%%%%%%%%%%%%%%%%%%%%%%%%%%%
% April 1996                     Petr Ol\v{s}\'ak

% The information about using the macro code128.tex is included here.
% You can try to TeX all examples presented here with plain TeX format.
%
% The Code 128 standard serves encoding (and decoding) the input string into
% barcode. The input string can be arbitrary long and it can contain 128
% different characters:
%
% - Digits: 0123456789
% - Uppercase and lowercase letters
% - Other ASCII characters: ! ' ( ) * + , - . / : ; < = > ? @ [ ] ` |
% - TeX special characters (must be escaped): \# \$ \% \& \\ \^ \_ \{ \} \~
% - Special control characters from ASCII table are implemented as control
%   sequences: \NUL \SOH \STX \ETX \EOT \ENQ \ACK \BEL \BS \HT \LF \VT \FF
%   \CR \SO \SI \DLE \DCone \DCtwo \DCthree \DCfour \NAK \SYN \ETB \CAN \EM
%   \SUB \ESC \FS \GS \RS \US \DEL
%
% After you \input the macro code128.tex by:

\input code128

% you can use the macro "\code". This macro encodes string as parameter
% enclosed in braces and produces the \hbox with barcode. For example:

\code{text}              \bigskip % (for vertical space from next example)

% produces the barcode of string "text". The spaces in input string are not
% significant. That is, the \code{t e x t} produces the same result.
% There is an exeption: Before closing brace the space is prohibited.
% For instance: \code{text } produces an error.
% If you want to encode the spaces too, you have to express the spaces by
% control sequence "\ " or "\SP". For example:

\code{text\ with\ spaces}         \bigskip

% Remember, you must escape all TeX special characters:

\code{\{ \% \# \^}                \bigskip

% The special ASCII characters can be encoded by control sequnces:

\code{abc\ESC ABC\\\CR}          \bigskip

% If you wish to add the printed version of string bellow barcode, you can use
% the "\codetext" alternative. It works in the same way as "\code", exept the
% output box is \vbox contained the \hbox with barcode and \tt version of
% input text (centered). For example

\codetext{text}                   \bigskip

% produces the same as in  previous example plus {\tt text}.
% If you use the control characters in input string and use the \codetext,
% you have to define the control sequences before using them. For example:

\def\\{\char`\\ }
\def\ESC{$\langle${\rm ESC}$\rangle$}
\def\CR{$\langle${\rm CR}$\rangle$}

\codetext{abc\ESC ABC\\\CR}                       \bigskip

% If you don't define them, an error "undefined control sequence" may be
% occured.
%
% If you wish to label the code by another string, then coded string, you
% can use the "\codeothertext" macro with two parameters. First parameter is
% the same as in \code command and second parameter will be printed by \tt.
% For example:

\codeothertext{01 70 08 39 90 31 99 62 07 69 36 20 10}%
              {0170-0839(199602)36:2;1-A}                   \bigskip

% You can change three dimension parameters for your special purposes:
% The implicit values follows:
% \X=.33mm         % The X module (all width dimensions are multiples of \X)
% \barheight=1.5cm % The code height.
% \bcorr=.020mm    % reduction (bars are thinner by \bcorr and spaces taller)
% Don't change the \bcorr parameter unless you know the tolerances of
% barcode and ink behavior of your output device. The \bcorr=.020mm is good
% value for offset printing.

% For example, we change:

\X=.5mm        % The width of code will be greater
\barheight=1cm % the height will be smaller.

\codetext{01 70 08 39 90 31 99 62 07 69 36 20 10}        \bigskip

% Notice1: The algorithm for encoding is not explicit. The results from
% different softare may be differ. You cannot debug the software only by
% comparing the results with other software. It is necessary to try the
% decoder for debugging the soft. If the output from decoder is the same as
% input string, all is OK. For more information about algorithm of Code 128
% standard see the end of file code128.tex.

% Notice2: You can use the macro on your own risk. There is no warranty
% that the macro works OK because author has no possibility to try any
% decoder. If you find out any bug, please report this (including the
% *.log file and the desription of the problem) to author's address:
% <olsak@math.feld.cvut.cz>.

% Some other examples (for testing).

\codetext{0170012233abc80911} \bigskip

\codetext{\CR\ESC ab\ESC de\#fgABC}

\end
