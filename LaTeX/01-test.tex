\documentclass[14pt]{extreport}

\usepackage{ifxetex}                      %% Для сборки документа и pdflatex'ом, и xelatex'ом

\ifxetex
    %% xelatex
    \usepackage{polyglossia}                       %% загружает пакет многоязыковой вёрстки
    \setdefaultlanguage[spelling=modern]{russian}  %% устанавливает главный язык документа
    \setotherlanguage{english}                     %% объявляет второй язык документа
    \defaultfontfeatures{Ligatures={TeX}}          %% свойства шрифтов по умолчанию
    \setmainfont[Ligatures={TeX}]{CMU Serif}       %% задаёт основной шрифт документа
    \setsansfont{CMU Sans Serif}                   %% задаёт шрифт без засечек
    \setmonofont{CMU Typewriter Text}              %% задаёт моноширинный шрифт
\else
    %% pdflatex
    \usepackage{cmap}                     %% Поиск русских  слов в pdf
    \usepackage[T2A]{fontenc}             %% Внутренняя кодировка шрифта
    \usepackage[utf8]{inputenc}           %% Кодировка исходного текста
    \usepackage[english,russian]{babel}   %% Поддержка русского текста
\fi

\usepackage{soulutf8}


\usepackage{geometry}
    \geometry
        {
        %paper=a4paper,
        paperwidth=58mm,
        paperheight=59mm,
        %margin=0.5mm,
        top=3mm,
        bottom=3mm,
        right=3mm,
        left=3mm,}
        %margin=0mm,
        %heightrounded, % rounding height in some strange way
        %showframe}

%\usepackage{showframe} % показывает рамки?



\usepackage[bottom]{footmisc}
\usepackage{pst-barcode,pstricks-add} % пакеты для штрихкодов, использовать вместе с пакетом auto-pst-pdf!
\usepackage{auto-pst-pdf} % использовать с pdflatex только с аргументом --shell-escape!

%\tolerance=1000 % увеличение разрешённых пробелов между словами
\sloppy % максимальное увеличение пробелов. ИСПОЛЬЗОВАТЬ С КРАЙНЕЙ ОСТОРОЖНОСТЬЮ!

%\setlength\parindent{1mm} % paragraph width
\def\text{TRN01-01}
\def\textloader{Егор Летов грузчик}
%\def\text{dummy text}
%\def\widthvar{2.0553} floating doesn't work?

\renewcommand{\familydefault}{\sfdefault} % изменение шрифта

\begin{document}
    \def\text{TRN-01-01}
    \centering
        \noindent % выключение отступа
        \begin{pspicture}%(0,0)(10,10) % idk what is that, size doesn't changing
            \psbarcode{\text}{width=2.0553 height=1.8}{code128}% width=0.5 height=1.5}{code128} % здесь задаётся размер, yeah!% height=4mm=4/25.3    in=0.158in % includetext guardwhitespace
        \end{pspicture}\\ % \\ is newline
        \textbf{\text}

    \def\text{TRN-01-01}
    \centering
        \noindent % выключение отступа
        \begin{pspicture}%(0,0)(10,10) % idk what is that, size doesn't changing
            \psbarcode{\text}{width=2.0553 height=1.8}{code128}% width=0.5 height=1.5}{code128} % здесь задаётся размер, yeah!% height=4mm=4/25.3    in=0.158in % includetext guardwhitespace
        \end{pspicture}\\ % \\ is newline
        \textbf{\text}
\end{document}

% size https://tex.stackexchange.com/a/62832/133780
% centering https://tex.stackexchange.com/questions/23650/when-should-we-use-begincenter-instead-of-centering
% shange font https://stackoverflow.com/questions/877597/how-do-you-change-the-document-font-in-latex
% bold font https://tex.stackexchange.com/questions/41681/correct-way-to-bold-italicize-text
% output +
% barcode +
% change size of barcode +
% change size of barcode in milimiters -
% centering of all +
% proper magrins of barcode +
% geometry +
% proper geometry +
% no blank pages +
% proper margins +
% proper fonts -
% variant of height for two lines of text -
% proper size of font - some type of that+
% a4 -
% print -
% codegen -
% --- python warning when barcode text longer 14 symbols

% один из вариантов задания размера
% set page dimensions
% \paperwidth  = 30cm
% \paperheight = 30cm
% use geometry pass option to pass dimensions to output driver (XeLaTeX)
% \usepackage[pass]{geometry}

%68
